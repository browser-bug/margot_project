\section{Introduction}


The main goal of mARGOt is to provide a dynamic autotuning framework, to enhance an application with an adaptation layer.
The mARGOt user guide, provides an high level description of the framework and example to help the integration process.
This document describes mARGOt heel, a collection of tools that aim at easing the integration process and the management of the application knowledge.
Before reading this document, we advise to go through the mARGOt user manual.


In particular, mARGOt heel is composed by two elements: the mARGOt command line interface (\textit{margot\_cli}) and the mARGOt high-level interface (\textit{margot\_if}).
margot\_cli is a tool written in python that provides utility feature to manage the application knowledge and create a simple Design Space Exploration script, based on \textit{make} files.
margot\_if is CMake-based library which generates an high-level interface of mARGOt, according to XML configuration files, using margot\_cli to generate the required glue code.
The main idea is that, provided the application requirements described in XML, it is possible to generate a simple interface, composed by few functions, to hide as much as possible implementation details of the autotuner framework.


This document is structured as follows, at first it describes the high level interface generated by margot\_if, then it describes the syntax of the related XML configuration files.
The last part of the document provides an overview of all the utility commands provided by margot\_cli.
For an example of integration, please refer to the tutorial repository on GitLab\footnote{\url{https://gitlab.com/margot_project/tutorial}}

