\section{How to start the remote application handler}

Since the AGORA application handler implementation requires several dependencies, by default is disabled.
It is possible to enable it, using the cmake flag \textit{-DWITH\_AGORA=ON}.
The build system will automatically download and compile the Cassandra C/C++ client and the Apache Paho MQTT client library.

The remote handler executable is named \textit{agora} and expose several flags to configure its behavior.
Please, use the \textit{--help} command option to have the full list.
In particular, it is possible to configure parameters of the MQTT connection, parameters to the storage connection and the parameters of AGORA itself.

For the MQTT connection, it is possible to specify the url of the broker, the username and password required to authenticate to the broker (if needed) and the communication quality of service.
For the storage conncetion, it is possible to specify the url of one cluster of the databse, and the usename and password required for the authentication (if needed).
You need to start the MQTT broker and the Cassandra database before of running the AGORA application handler.

The server requires only two parameters to successfully start its execution: the workspace folder and the plugin folder.
The workspace folder is the path where the application is able to store files to generate the application knowledge, while the plugin folder is the path to the available plugins.
The standard ones, shipped with AGORA, are located in \textit{<repository\_root>/agora/plugins}.
