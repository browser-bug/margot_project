\section{Remote application handler}

The remote application handler is designed to orchestrate the Design Space Exploration of an unknown application in a distributed fashion.
Which means that each instance of the application contributes at building the knowledge, exploring a different configuration, without restarting the server.
To achieve this goal, we need to define three concepts: how to explore the Design Space (i.e. the Design of Experiments), what is the application knowledge and how to predict the behavior of a metric of interest. 


\subsection{Design of Experiments}

Within this context, the Design Space Exploration (DSE) aims at exploring the Design Space to characterize the behavior of the application.
The Design of Experiments (DoE) aims at selecting which configuration must be explored.
Given the design space of each software knob of the target application, the most common DoE is exploring all the combinations (named full-factorial).
However, since the design space grows exponentially with the number of software knobs, and with the admissible values of each software knob, a full-factorial exploration might be unfeasible.
This is a well known topic in literature, therefore are available different techniques that aims at selecting the most meaningful configurations given the method used to generate the application knowledge.

The remote application handler, enable the user to select which DoE technique to use during the DSE.
To have a list of all the supported DoE in the current version of AGORA, please refer to help command.
Moreover, to produce a more robust knowledge of the application, it is possible to set the minimum number of times that each configuration must be observed.


\subsection{The application knowledge}

In the mARGOt user guide, the application knowledge was implicitly defined by the list of Operating Points.
Since we aim at modeling the application behavior at runtime, we define the application knowledge as the full-factorial combination of all the admissible values of all the software knobs and of the features of the application.

Therefore, the final goal of the AGORA framework is to produce a list that relates a configuration, for each possible feature, with the expected value of the metrics of interest.
In particular, it characterize a metric with its mean value and with its standard deviation.
Which is delivered to each client of the application.


\subsection{How to generate the application knoweldge}

During the DSE, the remote application handler dispatch configuration to explore to the available clients.
However, since the selected DoE may explore only a subset of the Design Space and since we can't force the exploration of data features, we need a technique to generate the application knowledge.

This topic is a well known topic in literature, therefore are available different techniques that aims at solving this problem.
Since each technique shines in a particular context, the remote application handler let the user to select its preferred technique, with metric granularity.
Moreover, since a metric might be totally application specific, e.g. the elaboration error, AGORA implements a plugin system that enable the user to write its own technique to derive a metric.
To have a list of all the supported prediction methods in the current version of AGORA, please refer to help command.
