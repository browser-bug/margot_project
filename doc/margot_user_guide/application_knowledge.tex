\section{Application knowledge}

The application knowledge describes the expected behavior of the application, in terms of its EFP of interest.
To represent the latter concept, mARGOt uses Operating Point objects.
An Operating Point is composed by two segments: the \textit{software knobs} segment and the \textit{metrics} segment.
The former represents a configuration of the application, the latter represents the performance reached by the application, using the related configuration.
The collection of several Operating Points define the application knowledge.
Which means that, any configuration provided by the autotuner, must be part of the application knowledge.


\subsection{Operating Point geometry}

mARGOt implements each segment of the Operating Point as an array of DataType.
The DataType is a type suitable to describe a value of either a configuration or a metric.
In particular, if the target field of the Operating Point has a known static value, we can represent such field using just its mean value.
For examples, the number of threads or the quality of the results may be fully described with a single number.
If the target field of the Operating Point is a measured metric, with a stochastic component, we must represent such field using (at least) its mean and standard deviation values.
For examples, the execution time of the kernel or its process CPU usage are a random variable, with a mean and a standard deviation.

\begin{figure}[!t]
	\centering
	\lstset{language=MyCPP}
	\begin{lstlisting}
	// define the Operating Point geometry
	using KnobsType = margot::OperatingPointSegment< 2, margot::Data<int> >;
	using MetricsType = margot::OperatingPointSegment< 2, margot::Distribution<float> >;
	using MyOperatingPoints = margot::OperatingPoint<KnobsType, MetricsType>
	
	// declare the application knowledge
	std::vector< MyOperatingPoints > application_knowledge = {
		{ // first Operating Point
			{1, 2},
			{margot::Distribution<float>(1, 0.1), margot::Distribution<float>(1, 0.1)}
		},
		{ // second Operating Point
			{2, 3},
			{margot::Distribution<float>(1, 0.1), margot::Distribution<float>(2, 0.1)}
		}
	};
	\end{lstlisting}
	\caption{Example of C++ code to define the application knowledge.}
	\label{fig:operating_point_examples}
\end{figure}

To improve the efficiency of the framework, the geometry (the C type) of the Operating Point must be known at compile-time.
In this way, mARGOt is able to exploit C++ features, to specialize the implementation of the framework according to the application knowledge.
\prettyref{fig:operating_point_examples} shows an example on how it is possible to define the application knowledge.
In particular, lines 1-4 defines the geometry of the Operating Point, while lines lines 7-16 defines the actual knowledge of the application.
In this example, the kernel has two software knobs, which values might be defined using only integer numbers (line 2).
Moreover, the application developer is interest on only two EFP, represented as a distribution of floats (line 3).
Finally, the global geometry of the Operating Point is defined in line 4.
The actual application knowledge might be represented using any STL container.
In this example, we have chosen a std::vector, brace-enclosed initialized lines(6-16).
For a full description of the Operating Point implementation, please refer to the Doxygen documentation.


\subsection{How to obtain the application knowledge}

The application knowledge is considered as an input of the framework, usually derived from a Design Space Exploration.
Since this is a well known topic in literature, the application developer is free to choose the most suitable approach to derive the application knowledge.
For example, if the design space is small, it is possible to perform a full-factorial Design of Experiment and evaluate all the possible configuration of the software knobs.
If the design space is too big, to perform an exhaustive search, it is possible to employ response surface modeling technique to derive the application knowledge.
For example, it is possible to evaluate only a subset of the design space and then interpolate the missing point.


