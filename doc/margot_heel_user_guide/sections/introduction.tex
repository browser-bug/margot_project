\section{Introduction}


The main goal of mARGOt is to provide a dynamic autotuning framework, to enhance an application with an adaptation layer.
The mARGOt user guide, provides an high level description of the autotuning framework and an integration process example.
This document describes mARGOt heel, which supports the autotuning framework by abstracting extra-functional concerns from implementation details.
Therefore, we recommend to go through the mARGOt user manual before reading this document.

The mARGOt heel main goal is to expose to the end-user a single, simple, and consistent interface that hides as much as possible the mARGOt implementation details, named high-level interface.
In particular, the end-user should state extra-functional concerns in configuration files, then mARGOt heel generates a C++ library that implements the high-level interface.

This document guides the end-user through the mARGOt integration using the high-level interface.
At first, we introduce the target application model and the high-level interface, highlighting the relation with configuration files.
Then, we specify the syntax and semantic of the content of each configuration files.
Eventually, we show the integration process from the building system point of view, focusing on CMake.

