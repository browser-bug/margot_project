\section{Knowledge configuration file}
\label{sec:knowledge}

The knowledge file describes the behavior of a single block of code of the application, by varying the configuration of the software knobs. 
Typically, the knowledge is the output of a Design Space Exploration and the most simple representation is as a list of Operating Points.
Each Operating Point relates a configuration with the performance of the observed block of code.

\begin{figure}
\lstset{language=XML}
\begin{lstlisting}
<?xml version="1.0" encoding="UTF-8"?>
<points xmlns="http://www.multicube.eu/" version="1.3" block="kernel1">
	
	<point>
		<parameters>
			<parameter name="param1" value="1"/>
			<parameter name="param2" value="100"/>
		</parameters>
		<system_metrics>
			<system_metric name="quality" value="100" standard_dev="1.0"/>
			<system_metric name="throughput" value="324.279" standard_dev="1.0"/>
		</system_metrics>
	</point>
	
	<point>
		<parameters>
			<parameter name="param1" value="5"/>
			<parameter name="param2" value="3"/>
		</parameters>
		<system_metrics>
			<system_metric name="quality" value="500" standard_dev="1.0"/>
			<system_metric name="throughput" value="17.487" standard_dev="1.0"/>
		</system_metrics>
	</point>

</points>
\end{lstlisting}
\caption{Example of a knowledge configuration file; it refers to the block \textit{kernel1} in \prettyref{fig:code_example1} (Knowledge1.xml)}
\label{code:knowledge_file}
\end{figure}

\prettyref{code:knowledge_file} shows an example of the syntax of the knowledge file.
In particular, the root XML element is the tag \textit{points}, which must define the attribute \textit{block}.
The latter is the name of the block of the application, described by the Operating Point list.
The name of the block must be a valid C++ namespace identifier.
The list is composed by a sequence of Operating Points, each one is represented with the tag XML \textit{point}.
The examples shows two different Operating Points (they have a different configuration of the software knobs).
Each Operating Point is composed by a pair of information: the configuration of the software knobs and the performance reached using such configuration.

In particular, the configuration is represented by the XML tag \textit{parameters}, while the value of each software knob is represented by the XML tag \textit{parameter}.
The latter XML element must define two attributes: the name of the target software knob (the attribute \textit{name}) and its actual value (the attribute \textit{value}).
The name of each software knobs must be an unique valid C++ enumerator.
The name is not case sensitive.

The performance of the block of code is represented by the XML tag \textit{system\_metrics}, while each metric of interest (that defines the performance) is represented by the XML tag \textit{system\_metric}.
The latter XML element must define two attributes: the name of the target metric (the attribute \textit{name}) and its actual value (the attribute \textit{value}).
Optionally, it may provide the standard deviation of the measure, using the XML tag \textit{standard\_dev}.
The name of each metric must be an unique valid C++ enumerator.
The name is not case sensitive.

\begin{figure}
\lstset{language=XML}
\begin{lstlisting}
<?xml version="1.0" encoding="UTF-8"?>
<points xmlns="http://www.multicube.eu/" version="1.3" block="foo">

	<dictionary param_name="compiler_flag">
		<value string="-O2" numeric="1" />
		<value string="-O3" numeric="2" />
	</dictionary>


	<point>
		<parameters>
			<parameter name="compiler_flag" value="1"/>
		</parameters>
		<system_metrics>
			<system_metric name="time" value="123"/>
		</system_metrics>
	</point>

</points>
\end{lstlisting}
\caption{Example of a knowledge configuration file with string values for software knobs}
\label{code:knowledge_file_dictionary}
\end{figure}

The attribute \textit{value} (and \textit{standard\_dev}) of each field of the Operating Point must be a numeric value.
If the user requires strings as a value of a software knob, then it must also define a dictionary that translates the string to a numeric value.
In particular, for each non-numeric software knob, the user must define a \textit{dictionary} XML element.
The dictionary is composed by a list of \textit{value} elements that must define two attributes: the string value (with the attribute \textit{string}) and its related numeric value (with the attribute \textit{numeric}).
\prettyref{code:knowledge_file_dictionary} shows an example of knowledge file, where the software knob ``compiler\_flag'' requires string values.
In the current version is not supported having string values for any metric of interest.


\begin{figure}
	\lstset{language=XML}
	\begin{lstlisting}
	<?xml version="1.0" encoding="UTF-8"?>
	<points xmlns="http://www.multicube.eu/" version="1.3" block="foo">
	
	
		<point>
			<parameters>
				<parameter name="compiler_flag" value="1"/>
			</parameters>
			<system_metrics>
				<system_metric name="time" value="123"/>
			</system_metrics>
			<features>
				<feature name="size_x" value="100"/>
				<feature name="size_y" value="10" />
			</features>
		</point>
	
	</points>
	\end{lstlisting}
	\caption{Example of a knowledge configuration file with data features}
	\label{code:knowledge_file_data_feature}
\end{figure}

If the application knowledge takes into account features of the input, it is possible to enrich the definition of the Operating Points with that information.
In particular, all the input features of each Operating Point are represented by the XML tag \textit{features}.
Each data feature is represented by the XML tag \textit{feature}; the attribute \textit{name} represents the name of the data feature, while the attribute \textit{value} represents the numerical value of the feature.
