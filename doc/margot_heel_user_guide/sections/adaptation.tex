\section{Adaptation file specification}
\label{sec:adaptation}

\begin{figure}
\lstset{language=json}
\begin{lstlisting}
{
	"name": "bar",
	"version": "4.0",
	"blocks":
	[
		{ "name": "block1", ... },
		{ "name": "block2", ... }
	]
}
\end{lstlisting}
\caption{The top-level description of the apllication extra-functional cencerns.}
\label{code:min_adaptation}
\end{figure}

This section describes in details the syntax and semantic of the configuration file that states the extra-functional concerns.
In particular, it follows at top-down approach, where the root element represents the application and it contains all the block descriptions.
\prettyref{code:min_adaptation} shows an example of a configuration file skeleton.
The root element must define the application name and version, using the tag \textbf{name} and \textbf{version}.
The \textbf{blocks} element contains the list of all the block managed by margot.
In this example, we have a block named ``block1'' and a second one named ``block2''.
The name of the block is represented by the element \textbf{name} and it must be a valid C++ identifier.
The remainder of the section will expand the block description, defining all its sub-elements.



\subsection{How to define monitors}
\label{ssec:monitor_element}


All the monitors that belong to a block are defined as a list contained in the \textbf{monitors} element.
Each monitor is defined by the following elements:
\begin{itemize}
	\item[name]: the monitor's name, it must be a valid C++ identifier and it must be unique.
	\item[type]: the monitor's type. \prettyref{appendix:monitor_implementation} provides the list of monitors shipped with mARGOt. If the type is arithmetic type, it will create a custom monitor of that type.
	\item[log]: the list of statistical properties that we want to display in the log. Each element of the list must assume one of the following values: \textbf{average}, \textbf{standard\_deviation}, \textbf{max}, and \textbf{min}.
	\item[constructor]: the list of parameters that the monitor's constructor is expecting. The order of the parameters must match the constructor signature. If we want to override the default parameters, we need to specify all of them. \prettyref{appendix:monitor_implementation} provides more details on constructor parameters for the known monitors. Custom monitors have a single constructor parameter: the observation window size.
	\item[start]: the list of parameters for starting a mesure. In the current version, no monitors requires any parameter to start a measure.
	\item[stop]: the list of parameters for stopping a measure. In the current version, only the throughput monitor and the custom monitor requires a stop parameter, which are the amount of data elaborated and the value to push in the monitor, respectively.
\end{itemize}

All the parameters specified in the \textit{constructor}, \textit{start}, and \textit{stop} fields can be an immediate or a variable.
An immediate parameter is a string that it is used ``as it is'', therefore it is set a compile time.
A variable parameter is a pair of strings that represents a C++ variable that will be exposed to the user through as parameter of the high-level interface, therefore it can change at runtime.
In particular, the first value of the variable parameter is its C++ type, while the second is its name.
Therefore, the latter must be a valid C++ identfier.
Moreover, to ensure the uniqueness of the variable parameters exposed in the high-level interface, the actual name of the parameter exposed through the high-level interface is ``<monitor\_name>\_<parameter\_name>''.

\begin{figure}
\lstset{language=json}
\begin{lstlisting}
"monitors":
[
	{
		"name": "error_monitor",
		"type": "float",
		"log": [ "average" ],
		"constructor": [ 22 ]
		"stop": [ {"error":"float"} ]
	},
	{
		"name": "time_monitor",
		"type": "time",
		"log": [ "average" ],
		"constructor": [ "margot::TimeUnit::MICROSECONDS", 1 ]
	}
]
\end{lstlisting}
\caption{This is an example on how we can define of two monitors}
\label{code:monitors}
\end{figure}

\prettyref{code:monitors} shows an example that defines an error monitor and a time monitor.
The error monitor will store $22$ elements of type \textit{floats}, to display the average in the logs.
Each observation is pushed in the monitor using a \textit{float} variable named ``error''.
This variable will be exposed to the user as parameter of the \textbf{push\_custom\_monitor\_values} high-level function.
According to generation convention, the name of the parameter will be ``\textit{error\_monitor\_error}''.
The second monitor observe the elapsed time between the \textbf{start\_monitors} and \textbf{stop\_monitors} high-level functions.
Since all its parameters are immediate, they will be compile-time constant in the generated code.



\subsection{How to define software-knobs}
\label{ssec:knob_element}


All the block software-knobs are defined as a list contained in the \textbf{knobs} element.
Each software-knob is defined by the following elements:
\begin{itemize}
	\item[name]: the knob name, it must be a valid C++ identifier.
	\item[type]: the C++ type of the knob. It can be an arithmetic type or a string. If the type is integral, it will be aliased with a fixed size. For example, ``int'' will be automatically aliased to ``int32\_t'' in certain systems. If the type is a string, then it will be considered as an enumeration. Therefore, it must be specified in the configuration file all the possible values that the knob can take, using the \textbf{values} field.
	\item[values]: the list of all the possible values that the knob may take. This field is mandatory if we plan to learn the application knowledge at runtime or if we the know type is a string.
	\item[range]: it's a list composed of two or three values (\textit{min}, \textit{max}, and \textit{step}) that generates all the values in the range [\textit{min}; \textit{max}) with step \textit{step}. The parameter \textit{step} is optional and its default value is $1$. It replaces the \textbf{values} field.
\end{itemize}

The software-knobs are part of the application geometry, along with the definition of the metrics of interest.
They are optional, meaning that we can omit this element in a block defition.
However, in this case we are not able to tune the application, but we can use the mARGOt facilities to monitor its performance.
If we define the software-knobs, we need to define also the metrics of interest.


\begin{figure}
\lstset{language=json}
\begin{lstlisting}
"knobs":
[
	{
		"name": "threads",
		"type": "int",
		"range": [ 1, 33, 1 ]
	},
	{
		"name": "algorithm",
		"type": "string",
		"values": [ "one", "two", "three" ]
	}
]
\end{lstlisting}
\caption{This is an example on how we can define of two software-knobs}
\label{code:knobs}
\end{figure}

\prettyref{code:knobs} shows an example that defines two software-knobs, named ``threads'' and ``algorithm''.
Since the first one represents the number of threads used in an application, its type is integral.
Moreover, if we assume that the target platforms has $32$ hardware thread, its range spans from $1$ thread to $32$, with step $1$.
The second software-knob represents the name of the algorithm to use in the elaboration.
Therefore, it defines the enumeration of all the possible values that the knobs can have.



\subsection{How to define metrics of interest}
\label{ssec:metric_element}


All the metrics of interest are defined as a list contained in the \textbf{metrics} element.
Each metric is defined by the following elements:
\begin{itemize}
	\item[name]: the metric name, it must be a valid C++ identifier.
	\item[type]: the C++ type of the knob. It must be an arithmetic type. If the type is integral, it will be aliased with a fixed size. For example, ``int'' will be automatically aliased to ``int32\_t'' in certain systems.
	\item[distribution]: tells whether this metric is determistic or not. Which means that if the value is true, we need a mean and a standard deviation to define a value.
	\item[observed\_by]: relate this metric to a monitor. This field is required if we need to adapt reactively of if we need to learn the application knowledge at runtime.
	\item[reactive\_inertia]: the size of circular buffer that adjust the expected metric value, according to the runtime observation. Larger values lead to a slower reaction. If the value is zero, it means that we don't want to adapt reactively at runtime. This field is optional and its default value is zero.
	\item[prediction\_plugin]: the name of the plugin in charge of predicting this metric. This field is meaningful only if we learn the application knowledge at runtime.
	\item[prediction\_parameters]: a list of key-value parameters that we would like to forward to the prediction plugin. This field is meaningful only if we learn the application knowledge at runtime.
\end{itemize}

The metrics of interest are part of the application geometry, along with the definition of the software-knobs and input features.
They are optional, meaning that we can omit this element in a block defition.
However, in this case we are not able to tune the application, but we can use the mARGOt facilities to monitor its performance.
If we define the metrics of interest, we need to define also the software-knobs.

\begin{figure}
\lstset{language=json}
\begin{lstlisting}
"metrics":
[
	{
		"name": "throughput",
		"type": "float",
		"distribution": "yes",
		"observed_by": "throughput_monitor",
		"reactive_inertia": 5,
		"prediction_plugin": "hth",
		"prediction_parameters":
		[
			{"param1": "value1"},
			{"param2": "value2"}
		]
	},
	{
		"name": "quality",
		"type": "float",
		"observed_by": "quality_monitor",
		"prediction_plugin": "hth"
	}
]
\end{lstlisting}
\caption{This is an example on how we can define of two software-knobs}
\label{code:metrics}
\end{figure}



\prettyref{code:metrics} shows an example that defines two metrics, named ``throughput'' and ``quality''.
In this example we consider the throughput as distribution since the measure depends on several factors that are unknown at priori and that depends on the system evolution.
Therefore, we want to react on those changes, but with some inertia to lower the effect of noise.
To predict this metric we use the plugin ``hth'', where we define two additional parameters to improve the prediction.
Moreover, in this example we consider the error determistic, therefore we can describe its values using only the mean.
Since computing the error might be expensive to perform at runtime, we don't adapt reactively.
Indeed, we use the measurement only to learn its behaviour at runtime, using the plugin ``hth''.




\subsection{How to define input features}
\label{ssec:feature_element}


The input features are defined by two elements.
The \textbf{feature\_distance} element which represent how mARGOt should compute the distance between two input features.
This element can have only two values: \textit{euclidean} or \textit{normalized}.
The \textbf{features} element stores the list of all the fields that compose the input features.
Each field is defined by the following elements:
\begin{itemize}
	\item[name]: the knob's name, it must be a valid C++ identifier.
	\item[type]: the C++ type of the knob, it must be an arithmetic type.
	\item[comparison]: the type of constraint that we would like to impose for the cluster selection. It is possible to use one of the following values:
	\begin{itemize}
		\item[le] to express the ``less or equal than'' constraint
		\item[ge] to express the ``greater or equal than'' constraint
		\item[-] if we don't care about imposing a constrant. This is the default value.
	\end{itemize}
\end{itemize}

The input features are part of the application geometry, along with the definition of the metrics of interest and software-knobs.
Differently from the former sections, the input feature definition is not mandatory.
However, if we define input features, we need to define also the metrics of interest and software-knobs.



\begin{figure}
\lstset{language=json}
\begin{lstlisting}
"feature_distance":"euclidean",
"features":
[
	{
		"name": "f1",
		"type": "int",
		"comparison": "le"
	}
]
\end{lstlisting}
\caption{This is an example on how we can define input features}
\label{code:features}
\end{figure}


\prettyref{code:features} shows an example that defines the input features as composed by a single integer field named ``f1''.
We impose a constraint on the selection of the input feature cluster.
In particolar, we state that the feature \textit{f1} of a cluster must be lower or equal than the feature of the current input to be selected.
Moreover, we stated that we want to select the closest one, among the eligble ones, computing the euclidean distance.





\subsection{How to learn the application knowledge at runtime}
\label{ssec:agora_element}


If we want to learn the application knowledge at runtime, we need to connect to an MQTT broker and provide information about DoE and clustering (if any).
All these information are contained in the \textbf{agora} element, where we can specify the following information as inner elements:
\begin{itemize}
	\item[borker\_url]: the string containing the url of the target broker in the form \textit{<protocol>://<url>:<port>}. The protocol can be either \textit{tcp} or \textit{ssl}; and the port usually is $1883$ for tcp or $8883$ for ssl. Both, the protocol and port can be omitted. The default value depends on the MQTT driver implementation.
	\item[broker\_username]: the username used to log in the MQTT broker, leave empty if not required.
	\item[broker\_password]: the password used to log in the MQTT broker, leave empty if not required.
	\item[broker\_qos]: the amount of effort spent by MQTT to check if the receiver has received the sent message. It is an integral value in range [$0$,$2$], where with the value $2$ MQTT spend most effort.
	\item[broker\_ca]: the path to the broker certificate, leave empty if not required.
	\item[client\_cert]: the path to the client certificate, leave empty if not required.
	\item[client\_key]: the path to the key of the client certificate, leave empty if not required.
	\item[clustering\_plugin]: the name of the plugin in charge of clustering the observed input features. This field is meaningful only if we have input features.
	\item[clustering\_parameters]: a list of key-value parameters that we would like to forward to the clustering plugin. This field is meaningful only if we have input features.
	\item[doe\_plugin]: the name of the plugin in that defines the DoE to explore at runtime.
	\item[doe\_parameters]: a list of key-value parameters that we would like to forward to the DoE plugin.
\end{itemize}


Beside the information required to connect with the MQTT broker, all the other information are forwarded to the remote component in charge of orchestrating the DSE, without validating them.
If the connection with broker fails, the generated code rise an \textit{std::runtime\_error}, since the specification of the \textbf{agora} component is required only if we want to learn the application knoweldge and it can't happen without a MQTT broker.


\begin{figure}
\lstset{language=json}
\begin{lstlisting}
"agora":
{
	"borker_url": "127.0.0.1:1883",
	"broker_username": "margot",
	"broker_password": "margot",
	"broker_qos": 2,
	"broker_ca": "",
	"client_cert": "",
	"client_key": "",
	"clustering_plugin": "clusty",
	"clustering_parameters":
	[
		{"algorithm": "kmeans"},
		{"number_centroids": 5}
	],
	"doe_plugin": "jhondoe",
	"doe_parameters":
	[
		{"constraint": "knob1 + knob2 < 40"}
	]
}
\end{lstlisting}
\caption{This is an example on how we can define the information required to learn the application knowledge at runtime}
\label{code:agora}
\end{figure}


\prettyref{code:agora} shows an example where mARGOt will connect locally to the MQTT broker, with authentication.
Since we are communicating locally, encription is not required.
Moreover, we forward two parameters to the clustering pluging, while we forward a single parameter.




\subsection{How to define the optimization problem}
\label{ssec:agora_element}

The element that contains the list of optimization problem that defines the application requirements is \textbf{extra-functional\_requirements}.
Each optimization problem is defined by a name (using the element \textbf{name}), the objective function (named rank in mARGOt context), and the list of constraint.

The optimization function is defined by the element \textbf{maximize} or \textbf{minimize}, according to the rank direction.
The only child element of the rank direction is how the user would combine the different rank fields, using either the element \textbf{geometric\_mean} or \textbf{linear\_mean}.
The child of the latter element is a list, composed of rank field.
Each rank field is defined by a key-value element.
The key is the name of the related metric or software-knob.
The value is its coefficient in the formula combination.



The list of constraint is represented in the \textbf{subject\_to} element, where the order in which they appear in the configuration file set the priority between them.
In particular the constraint on the top has higher priority of the one in the bottom.
Each constraint is defined by the following fields:
\begin{itemize}
	\item[subject]: the name of the target metric or software-knob.
	\item[comparison]: the kind of comparison used in this contraint. The semantic is that \textit{<field>} must be \textit{<operator} than \textit{value}.
	\item[value]: the initial value of the constraint. It can be updated at runtime.
	\item[confidence]: the number of times that we need to consider the standard deviation of the Operating Point. If we want to consider only the mean value, we can set the confidence to zero.
\end{itemize}


The specification of the optimization problem happens during the initialization of the data structures.
If we want to update the value of a constraint, we can do it by setting the value of the related goal.
In particular, the related goal is automatically generated and its name is \textit{<name>\_constraint\_<index constraint>}.
For example the following instruction set the goal on the second constraint of the optimization problem named ``problem'', in the block named ``elaboration''.
\begin{lstlisting}
margot::elaboration::context().goals.problem_constraint_1.set(25);
\end{lstlisting}
Moreover, if we define more than one optimization problem, the application is in charge of selecting the starting one.
By default it will be the last problem defined in the configuration file.


\begin{figure}
\lstset{language=json}
\begin{lstlisting}
"extra-functional_requirements":
[
	{
		"name":"default",
		"maximize":
		{
			"geometric_mean":
			[
				{"quality": 2},
				{"throughput": 5}
			]
		},
		"subject_to":
		[
			{
				"subject":"throughput",
				"comparison": "ge",
				"value": 25.0,
				"confidence": 3
			}
		]
	}
]
\end{lstlisting}
\caption{This is an example on how we can define the information required to learn the application knowledge at runtime}
\label{code:agora}
\end{figure}


\prettyref{code:agora} shows an example where we define a single optimization problem named ``default''.
In terms of mARGOt representation, the target problem is the following:
\begin{equation}
\label{eq:optimization-general}
\begin{array}{rrclcl}
\displaystyle \max & \multicolumn{4}{l}{E(quality)^{2} \times E(throughput)^{5}} \\
\textrm{s.t.} & C_0: E(throughput) - 3 \times \sigma_{throughput}  & \leq   & 25
\end{array}
\end{equation}
where the mean and standard deviation of the \textit{quality} and \textit{throughput} may change if we change software-knobs configuration.










